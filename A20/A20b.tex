\documentclass[a4paper,headsepline,11pt,DIV15,twoside,titlepage,tablecaptionabove,bibtotocnumbered]{scrartcl}
%bibtotocnumbered für Numerierung der Literatur
\usepackage[utf8]{inputenc}
\usepackage{color}
\usepackage{graphicx}
\usepackage[ngerman]{babel}
\usepackage[misc]{ifsym} 			%for letter icon on titlepage
\usepackage{pdfpages}
\usepackage{amsmath,amsthm,amssymb}	
\usepackage{geometry}				%fuer Silbentrennung
\usepackage{scrpage2}				%fure pagestyle
\usepackage{parskip}				%Text wird eingerückt nach Bild und Section: Unterdrückung
%\usepackage{float}
\usepackage[
	colorlinks=true,	%blaues inhaltsverzeichniss
	linkcolor=blue
]{hyperref}
\usepackage[all]{hypcap}			%Verlinkungen genau auf Abb. und 								nicht auf Caption
\usepackage{bold-extra}
\usepackage{cite}
\usepackage{svg}
%\usepackage{siunitx}	%Verwendung von Abstand zw Einheiten etc: Nachschlagen, ist nützlich
\usepackage{subfig}
\usepackage{bm}
\usepackage{todonotes}
\usepackage{caption}
\usepackage{todonotes}

\geometry{
	left=3.0cm,
	right=2.5cm,
	top=2.5cm,
	bottom=2.5cm
}
\date{29.04.19}
\pagestyle{scrheadings}
\ohead{Julian Hoßbach \\ Simon Koppenhöfer}
\chead{Gruppe: A-086 / A-099 \\ Physikalisches Praktikum 1 Teil 2}
\ihead{Versuch: A20 \\ Datum: 29.04.2019}

\begin{document}
\clearpage

\begin{titlepage}
	\title{A20 - Frank-Hertz-Versuch}							%!!
	\titlehead{Praktikumsprotokoll Physikalisches Praktikum 2, 
			Universität Stuttgart}
	\author{ 
		\normalsize
		\begin{tabular}{lll}
		Verfasser: & Julian Hoßbach & Simon Koppenhöfer \\
		Matrikelnummer: & 3328323 & 3325782 \\
		\hline
		Gruppennummer: & A-086 & A-099\\
		Versuchsdatum: & 29.04.2019 &\\		%!!
		\hline
		Betreuer: & Christian Hölzel &\\		%!!
		\end{tabular}
		}
\end{titlepage}

\maketitle

 
\thispagestyle{empty}
\newpage

\tableofcontents
\addtocontents{toc}{\protect\enlargethispage{1cm}}


\newpage 
\thispagestyle{empty}
\quad 
\newpage
\setcounter{page}{1}

\newpage

\bibliography{texfiles/Bibliography}{}
\bibliographystyle{plain}

\end{document}
